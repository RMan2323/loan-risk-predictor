\documentclass{article}
\usepackage{graphicx} % Required for inserting images
\usepackage{float}
\usepackage{booktabs}
\usepackage{geometry}
\geometry{a4paper, margin=1in}

\title{K-Medoids Clustering Report}
\author{Aariz Shaikh}
\date{April 2025}

\begin{document}

\maketitle

\section{Results and Observations}

\subsection{K-Medoids Clustering}

Experiments were conducted using K-Medoids clustering on different variations of preprocessed datasets. The optimal number of clusters (K) was determined using the Silhouette score. Inertia was also calculated for each dataset.

\begin{table}[H]
\centering
\caption{Performance Metrics for Different Datasets using K-Medoids Clustering (K=2)}
\begin{tabular}{|c|l|c|c|}
\hline
\textbf{S.No} & \textbf{Dataset} & \textbf{Silhouette Score} & \textbf{Inertia} \\
\hline
0 & Train\_Scaled\_Cont\_OH.csv & 0.020905 & 91323.673402 \\
1 & Train\_MM\_OH.csv & 0.020905 & 91323.673402 \\
2 & train\_OH\_MM\_PCA15.csv & 0.071245 & 48832.189519 \\
3 & train\_OH\_MM\_PCA20.csv & 0.032095 & 57938.435981 \\
4 & train\_OH\_MM\_PCA25.csv & 0.032465 & 64564.495058 \\
5 & train\_OH\_MM\_PCA30.csv & 0.022067 & 71582.557634 \\
6 & train\_OH\_MM\_PCA35.csv & 0.020132 & 77578.418111 \\
7 & Train\_Scaled\_Cont\_Int.csv & 0.024165 & 70411.072568 \\
8 & Train\_MM\_Int.csv & 0.024165 & 70411.072568 \\
9 & train\_Int\_MM_PCA10.csv & 0.068794 & 36153.897856 \\
10 & train\_Int\_MM_PCA15.csv & 0.045488 & 46035.836064 \\
11 & train\_Int\_MM_PCA20.csv & 0.032297 & 55019.387744 \\
12 & train\_Int\_MM_PCA25.csv & 0.026684 & 62399.740681 \\
\hline
\end{tabular}
\label{tab:kmedoids_results}
\end{table}
\subsection{Best Datasets}
\subsubsection{Clustering Visualization}



\subsubsection{Clustering - Train\_Scaled\_Cont\_Int.csv}

\begin{figure}[H]
    \centering
    \includegraphics[width=0.8\textwidth]{Train_Scaled_Cont_Int_.png}
    \caption{Clustering - Train\_Scaled\_Cont\_OH.csv}
    \label{fig:_scaled_cont_int}
\end{figure}

\subsubsection{Clustering - Train\_MM\_Int.csv}

\begin{figure}[H]
    \centering
    \includegraphics[width=0.8\textwidth]{Train_MM_OH_Inertia.png}
    \caption{Inertia (Elbow Method) - Train\_MM\_OH.csv}
    \label{fig:inertia_mm_oh}
\end{figure}



\subsection{Silhouette Score Visualization}

The following figures show the Silhouette score plots for selected datasets with k=2,3,4, to determine the optimal number of clusters.

\subsubsection{Silhouette - Train\_Scaled\_Cont\_Int.csv}

\begin{figure}[H]
    \centering
    \includegraphics[width=0.8\textwidth]{Train_Scaled_Cont_Int_Silhouette.png}
    \caption{Silhouette Score - Train\_Scaled\_Cont\_Int.csv}
    \label{fig:silhouette_scaled_cont_Int}
\end{figure}

\subsubsection{Silhouette - Train\_MM\_Int.csv}

\begin{figure}[H]
    \centering
    \includegraphics[width=0.8\textwidth]{Train_MM_OH_Silhouette.png}
    \caption{Silhouette Score - Train\_MM\_Int.csv}
    \label{fig:silhouette_mm_Int}
\end{figure}



\subsection{Key Observations}

\begin{itemize}
    \item For all datasets, the best number of clusters (K) was found to be 2 based on the Silhouette score.
    \item Datasets preprocessed with PCA generally exhibited higher Silhouette scores, suggesting better-defined clusters.
\end{itemize}

\section{Conclusion}

K-Medoids clustering consistently identified 2 clusters as optimal across various preprocessed datasets, suggesting the presence of two distinct groupings within the data. The higher Silhouette scores observed with PCA-transformed datasets indicate improved cluster separation after dimensionality reduction.

\end{document}